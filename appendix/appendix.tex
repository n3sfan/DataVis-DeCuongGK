\section{Giải thích thuật ngữ}
\label{appendix:terminology}

Trong báo cáo này, một số thuật ngữ chuyên ngành tài chính và thị trường phái sinh được sử dụng. Dưới đây là phần giải thích chi tiết:

\subsection{Hợp đồng tương lai (Futures) và hợp đồng vĩnh cửu (Perpetual)}
\begin{itemize}
    \item \textbf{Hợp đồng tương lai:} Thỏa thuận mua hoặc bán một tài sản vào một thời điểm xác định trong tương lai với mức giá đã thỏa thuận hôm nay.
    \item \textbf{Hợp đồng vĩnh cửu:} Một dạng đặc biệt của hợp đồng tương lai nhưng không có ngày đáo hạn. Nhà đầu tư có thể giữ vị thế bao lâu tùy thích miễn là duy trì đủ ký quỹ.
\end{itemize}

\subsection{Vị thế Long và Short}
\begin{itemize}
    \item \textbf{Long (Mua):} Vị thế của nhà đầu tư kỳ vọng giá tài sản sẽ tăng. Họ mua hợp đồng với hy vọng bán lại ở giá cao hơn trong tương lai.
    \item \textbf{Short (Bán):} Vị thế của nhà đầu tư kỳ vọng giá tài sản sẽ giảm. Họ bán hợp đồng (mà họ có thể không sở hữu thực sự) với hy vọng mua lại ở giá thấp hơn để trả lại và hưởng chênh lệch.
\end{itemize}

\subsection{Maker và Taker}
\begin{itemize}
    \item \textbf{Maker (Người tạo thanh khoản):} Lệnh của họ không khớp ngay lập tức mà chờ người khác đến khớp. Họ cung cấp thanh khoản cho thị trường.
    \item \textbf{Taker (Người lấy thanh khoản):} Người đặt lệnh chốt ngay, khớp ngay lập tức với các lệnh đang chờ sẵn trong sổ lệnh. Họ lấy đi thanh khoản của thị trường.
\end{itemize}

\subsection{Thanh lý}
Trong giao dịch phái sinh, nhà đầu tư sử dụng đòn bẩy. Nếu giá thị trường di chuyển ngược lại với dự đoán của nhà đầu tư (giá giảm khi đang Long, hoặc giá tăng khi đang Short) đến mức số dư ký quỹ không còn đủ để duy trì vị thế, sàn giao dịch sẽ tự động đóng vị thế của nhà đầu tư để thu hồi vốn vay. Sự kiện này gọi là thanh lý.

\subsection{Hợp đồng mở (OI)}
Tổng số lượng các hợp đồng phái sinh đang lưu hành (chưa được tất toán hoặc thanh lý) trên thị trường tại một thời điểm cụ thể.
\begin{itemize}
    \item \textbf{OI tăng:} Dòng tiền mới đang đổ vào thị trường (nhiều vị thế mới được mở).
    \item \textbf{OI giảm:} Dòng tiền đang rút ra (các vị thế đang được đóng lại hoặc bị thanh lý).
\end{itemize}
\textbf{Lưu ý dữ liệu:} Trong báo cáo này, dữ liệu OI của Binance Coin-Margined được tính bằng số lượng hợp đồng, ví dụ 1 hợp đồng = 10 USD hoặc 100 USD tùy quy ước của sàn.
