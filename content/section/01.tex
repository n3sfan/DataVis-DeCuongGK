\section{Tên đề tài}
\textbf{Phân tích động lực thị trường và giá/thanh lý của ETH, meme coin (DOGE), alt-coin (SOL,...) trong giai đoạn quý 1 2024 (tháng 1/2024 - 3/2024)}

\section{Danh sách thành viên}
Được ghi ở trang bìa.

\section{Mô tả dữ liệu}
Dữ liệu được thu thập từ sàn giao dịch lớn là và Binance \cite{binancevision_futures_data}, ví dụ coin  Ethereum (ETH) trong các tháng 1 - 3/2024. Dữ liệu bao gồm ba nhóm chính, được mô tả chi tiết dưới đây.

Dữ liệu được lấy từ các sàn giao dịch phái si của Binance, thị trường hợp đồng tương lai (vĩnh cửu) của các coin được đề cập ở trên.


\subsection{Dữ liệu nến giá \cite{binance_klines}}
Dữ liệu này chứa thông tin về giá mở, cao, thấp, đóng và khối lượng giao dịch trong từng khung thời gian (OHLCV).

\begin{table}[H]
\renewcommand{\arraystretch}{1.25}
\centering
\begin{tabular}{|l|l|p{8cm}|}
\hline
\textbf{Tên trường} & \textbf{Kiểu dữ liệu} & \textbf{Mô tả chi tiết} \\ \hline
open\_time & Long & Thời gian mở nến (thời gian Unix). \\ \hline
open & Float & Giá mở cửa. \\ \hline
high & Float & Giá cao nhất trong phiên. \\ \hline
low & Float & Giá thấp nhất trong phiên. \\ \hline
close & Float & Giá đóng cửa. \\ \hline
volume & Float & Khối lượng giao dịch (số lượng hợp đồng - Contracts). \\ \hline
close\_time & Long & Thời gian đóng nến (thời gian Unix). \\ \hline
base\_asset\_volume & Float & Khối lượng giao dịch tính theo coin cơ sở (ETH). \\ \hline
count & Integer & Số lượng lượt khớp lệnh trong nến. \\ \hline
taker\_buy\_volume & Float & Khối lượng mua chủ động của Taker (Contract). \\ \hline
taker\_buy\_base\_asset\_volume & Float & Giá trị mua chủ động của Taker tính theo ETH. \\ \hline
\end{tabular}
\caption{Mô tả các trường dữ liệu Binance Kline}
\end{table}

\subsection{Dữ liệu lệnh thanh lý \cite{binance_liquidation}}
Dữ liệu này chứa thông tin chi tiết về từng lệnh thanh lý được kích hoạt.

\begin{table}[H]
\renewcommand{\arraystretch}{1.25}
\centering
\begin{tabular}{|l|l|p{8cm}|}
\hline
\textbf{Tên trường} & \textbf{Kiểu dữ liệu} & \textbf{Mô tả chi tiết} \\ \hline
time & Long & Thời gian khớp lệnh  \\ \hline
side & String & Chiều của lệnh thanh lý (SELL - Thanh lý lệnh Long, BUY - Thanh lý lệnh Short). \\ \hline
order\_type & String & Loại lệnh (thường là LIMIT). \\ \hline
time\_in\_force & String & Thời gian hiệu lực của lệnh (ví dụ: IOC - Immediate or Cancel). \\ \hline
original\_quantity & Float & Khối lượng gốc của lệnh thanh lý. \\ \hline
price & Float & Giá đặt của lệnh thanh lý (thường là giá phá sản). \\ \hline
average\_price & Float & Giá khớp trung bình thực tế. \\ \hline
order\_status & String & Trạng thái lệnh (ví dụ: FILLED - Đã khớp hoàn toàn). \\ \hline
last\_fill\_quantity & Float & Khối lượng khớp ở lần gần nhất. \\ \hline
accumulated\_fill\_quantity & Float & Tổng khối lượng đã khớp của lệnh này. \\ \hline
\end{tabular}
\caption{Mô tả các trường dữ liệu lệnh thanh lý}
\end{table}

\subsection{Dữ liệu các chỉ số khác \cite{binance_metrics}}

\begin{table}[H]
\renewcommand{\arraystretch}{1.25}
\centering
\begin{tabular}{|l|l|p{8cm}|}
\hline
\textbf{Tên trường} & \textbf{Kiểu dữ liệu} & \textbf{Mô tả chi tiết} \\ \hline
create\_time & DateTime & Thời gian ghi nhận dữ liệu (ví dụ: 2026-01-26 00:00:00). \\ \hline
symbol & String & Mã hợp đồng phái sinh (ví dụ: ETHUSD\_PERP). \\ \hline
sum\_open\_interest & Float & Tổng hợp đồng mở tính theo số lượng hợp đồng. \\ \hline
sum\_taker\_long\_short\_vol\_ratio & Float & Tỷ lệ khối lượng giao dịch Taker phe mua (Long) so với phe bán (Short). \\ \hline
\end{tabular}
\caption{Mô tả các trường dữ liệu chỉ số khác của phái sinh Binance}
\end{table}

\begin{figure}[H]
\centering
\includegraphics[width=\textwidth]{img/oi.png}
\caption{Biểu đồ giá, hợp đồng mở, khối lượng giao dịch}
\end{figure}

\section{Các câu hỏi phân tích dự kiến}
Nhóm đề xuất các câu hỏi phân tích sau:

\begin{itemize}
    \item \textbf{Câu hỏi 1:} Trong giai đoạn trên, xu hướng giá ETH biến động như thế nào so với các đường trung bình trượt giá đóng 7 cây nến?
    
    \begin{figure}[H]
    \centering
    \includegraphics[width=0.8\textwidth,height=0.8\textheight]{img/candle.png}
    \caption{Biểu đồ cây nến, trung bình trượt giá ETH}
    \end{figure}
    
    \item \textbf{Câu hỏi 2:} Mức độ tương quan giữa biến động giá của các coin diễn ra như thế nào trong các giai đoạn thị trường hưng phấn hoặc hoảng loạn?
    
    \item \textbf{Câu hỏi 3:} Tại các thời điểm xuất hiện "nến đỏ dài" (giá giảm mạnh trong thời gian ngắn), hợp đồng mở và khối lượng thanh lý thay đổi như thế nào? Có phải áp lực thanh lý lệnh "Long" là nguyên nhân chính dẫn đến đà giảm sâu?
    
    \begin{figure}[H]
    \centering
    \includegraphics[width=\textwidth]{img/liq-new.png}
    \caption{Biểu đồ lượng thanh lý của các coin, hình chữ nhật càng lớn = lượng thanh lý càng cao}
    \end{figure}


    \item \textbf{Câu hỏi 4:} Khối lượng thanh lý, hợp đồng mở theo thời gian như thế nào trong một "ngày sập giá", theo dõi?
    
    \begin{figure}[H]
    \centering
    \includegraphics[width=\textwidth]{img/liq-heat.png}
    \caption{Biểu đồ Heatmap khối lượng thanh lý trong ngày sập giá của ETH}
    \end{figure}
    
    \item \textbf{Câu hỏi 5:} Liệu có thể \textbf{dự đoán} các "trạng thái thị trường" hưng phấn, hoảng loạn (sập giá) dựa trên sự kết hợp giữa mất cân bằng thanh lý (khối lượng thanh lý lệnh Long $\gg$ Short hoặc ngược lại), tăng/giảm đột biến của hợp đồng mở, giá trong dữ liệu trên?
\end{itemize}
